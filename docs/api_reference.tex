\documentclass{article}
\usepackage{hyperref}
\usepackage{listings}
\usepackage{color}

\definecolor{codegreen}{rgb}{0,0.6,0}
\definecolor{codegray}{rgb}{0.5,0.5,0.5}
\definecolor{codepurple}{rgb}{0.58,0,0.82}

\lstdefinestyle{mystyle}{
    commentstyle=\color{codegreen},
    keywordstyle=\color{magenta},
    stringstyle=\color{codepurple},
    basicstyle=\ttfamily\small,
    breakatwhitespace=false,
    breaklines=true,
    captionpos=b,
    keepspaces=true,
    numbersep=5pt,
    showspaces=false,
    showstringspaces=false,
    showtabs=false,
    tabsize=2
}

\lstset{style=mystyle}

\title{ADAM API Reference}
\author{AOx0}
\date{\today}

\begin{document}

\maketitle

\section{Introduction}
This document provides detailed information about ADAM's API endpoints, interfaces, and integration points.

\section{REST API Endpoints}

\subsection{Firewall Management}
\subsubsection{GET /api/v1/rules}
Retrieves all current firewall rules.

\begin{lstlisting}[language=json]
// Request
GET /api/v1/rules

// Response
{
    "rules": [
        {
            "id": "rule1",
            "action": "allow",
            "source_ip": "192.168.1.0/24",
            "destination_port": 80,
            "protocol": "tcp"
        }
    ]
}
\end{lstlisting}

\subsubsection{POST /api/v1/rules}
Creates a new firewall rule.

\begin{lstlisting}[language=json]
// Request
POST /api/v1/rules
{
    "action": "deny",
    "source_ip": "10.0.0.0/8",
    "destination_port": 443,
    "protocol": "tcp"
}

// Response
{
    "id": "rule2",
    "status": "created"
}
\end{lstlisting}

\subsection{Network Analysis}
\subsubsection{GET /api/v1/traffic}
Retrieves current traffic statistics.

\begin{lstlisting}[language=json]
// Request
GET /api/v1/traffic

// Response
{
    "bytes_in": 1024000,
    "bytes_out": 512000,
    "packets_in": 1000,
    "packets_out": 500,
    "active_connections": 50
}
\end{lstlisting}

\section{WebSocket API}

\subsection{Real-time Events}
\subsubsection{/ws/v1/events}
Subscribes to real-time system events.

\begin{lstlisting}[language=json]
// Connection
ws://localhost:8080/ws/v1/events

// Event Types
{
    "type": "rule_match",
    "data": {
        "rule_id": "rule1",
        "timestamp": "2024-01-10T12:00:00Z",
        "details": {
            "source_ip": "192.168.1.100",
            "destination_port": 80
        }
    }
}
\end{lstlisting}

\section{Command Line Interface}

\subsection{Rule Management}
\begin{lstlisting}[language=bash]
# List all rules
adam rules list

# Add a new rule
adam rules add --action allow --source 192.168.1.0/24 --port 80

# Delete a rule
adam rules delete rule1
\end{lstlisting}

\subsection{Traffic Analysis}
\begin{lstlisting}[language=bash]
# Show traffic statistics
adam traffic show

# Start packet capture
adam capture start --interface eth0 --filter "port 80"

# Export traffic report
adam traffic export --format json --output report.json
\end{lstlisting}

\section{Integration Examples}

\subsection{Python Client}
\begin{lstlisting}[language=python]
import adam_client

# Initialize client
client = adam_client.Client("http://localhost:8080")

# Get rules
rules = client.get_rules()

# Add rule
new_rule = {
    "action": "allow",
    "source_ip": "192.168.1.0/24",
    "destination_port": 80
}
client.add_rule(new_rule)
\end{lstlisting}

\subsection{Rust Client}
\begin{lstlisting}[language=rust]
use adam_client::Client;

# Initialize client
let client = Client::new("http://localhost:8080");

# Get rules
let rules = client.get_rules().await?;

# Add rule
let new_rule = Rule {
    action: Action::Allow,
    source_ip: "192.168.1.0/24".parse()?,
    destination_port: 80,
};
client.add_rule(new_rule).await?;
\end{lstlisting}

\section{Error Handling}

\subsection{HTTP Status Codes}
\begin{itemize}
    \item 200 - Success
    \item 400 - Bad Request
    \item 401 - Unauthorized
    \item 403 - Forbidden
    \item 404 - Not Found
    \item 500 - Internal Server Error
\end{itemize}

\subsection{Error Response Format}
\begin{lstlisting}[language=json]
{
    "error": {
        "code": "INVALID_RULE",
        "message": "Invalid rule format",
        "details": {
            "field": "source_ip",
            "reason": "Invalid CIDR notation"
        }
    }
}
\end{lstlisting}

\end{document} 
