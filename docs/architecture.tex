\documentclass{article}
\usepackage{hyperref}
\usepackage{graphicx}
\usepackage{tikz}
\usepackage{listings}

\title{ADAM System Architecture}
\author{AOx0}
\date{\today}

\begin{document}

\maketitle

\section{Introduction}
This document provides a detailed overview of the ADAM system architecture, its components, and how they interact with each other.

\section{System Components}
ADAM consists of three main components:

\subsection{Firewall Component}
The firewall component is responsible for network packet filtering and security. It includes:
\begin{itemize}
    \item eBPF-based packet filtering
    \item Userspace control program
    \item Rule management system
\end{itemize}

\subsection{Network Parser (netp)}
The network parser component provides packet analysis capabilities:
\begin{itemize}
    \item Packet capture and analysis
    \item Protocol parsing
    \item Traffic monitoring
\end{itemize}

\subsection{Controller}
The controller component coordinates the system:
\begin{itemize}
    \item Management of firewall rules
    \item System configuration
    \item Component coordination
\end{itemize}

\section{Component Interactions}
\subsection{Data Flow}
The data flows through the system as follows:
\begin{enumerate}
    \item Network packets are intercepted by the eBPF program
    \item Packets are analyzed by the network parser
    \item The controller applies firewall rules based on analysis
    \item Packets are either allowed or dropped based on rules
\end{enumerate}

\section{Technical Details}
\subsection{eBPF Implementation}
The firewall uses eBPF for efficient packet processing:
\begin{itemize}
    \item XDP hooks for early packet processing
    \item TC (Traffic Control) filters for fine-grained control
    \item BPF maps for sharing data between kernel and userspace
\end{itemize}

\subsection{Network Parser Implementation}
The network parser is implemented with:
\begin{itemize}
    \item Zero-copy packet processing
    \item Protocol-specific parsers
    \item Efficient memory management
\end{itemize}

\section{Performance Considerations}
\subsection{Optimization Techniques}
The system employs several optimization techniques:
\begin{itemize}
    \item Lock-free data structures
    \item Zero-copy packet processing
    \item Efficient memory allocation
    \item Batch processing where applicable
\end{itemize}

\subsection{Resource Usage}
Typical resource usage patterns:
\begin{itemize}
    \item Memory footprint: 10-50MB depending on configuration
    \item CPU usage: 1-5% per core under normal load
    \item Network overhead: Minimal due to eBPF implementation
\end{itemize}

\section{Security Considerations}
\subsection{Security Features}
Built-in security features include:
\begin{itemize}
    \item Privilege separation
    \item Secure communication between components
    \item Input validation and sanitization
    \item Rate limiting and DoS protection
\end{itemize}

\section{Future Considerations}
Areas for potential future enhancement:
\begin{itemize}
    \item Additional protocol support
    \item Enhanced monitoring capabilities
    \item Integration with external security tools
    \item Performance optimizations
\end{itemize}

\end{document} 
