\documentclass{article}
\usepackage{hyperref}
\usepackage{listings}
\usepackage{enumitem}
\usepackage{color}

% Define Rust language for listings
\definecolor{rustcomment}{RGB}{0,128,0}
\definecolor{ruststring}{RGB}{128,0,0}
\definecolor{rustkeyword}{RGB}{0,0,255}

\lstdefinelanguage{rust}{
  keywords={
    self,mut,extern,unsafe,impl,fn,use,mod,pub,struct,enum,
    trait,where,match,if,else,for,loop,while,let,move,ref,
    type,const,static,box,super,return,true,false,in,continue,
    break,as,async,await
  },
  keywordstyle=\color{rustkeyword},
  sensitive=true,
  comment=[l]{//},
  morecomment=[s]{/*}{*/},
  commentstyle=\color{rustcomment},
  stringstyle=\color{ruststring},
  morestring=[b]",
  morestring=[b]',
  identifierstyle=\color{black},
}

\title{ADAM Testing Guide}
\author{AOx0}
\date{\today}

\begin{document}

\maketitle

\section{Introduction}
This guide provides comprehensive information about testing the ADAM system, including unit testing, integration testing, performance testing, and security testing.

\section{Testing Environment Setup}

\subsection{Development Environment}
\begin{itemize}
    \item Rust toolchain (nightly)
    \item Testing frameworks
    \item Mock services
    \item Test data generation tools
\end{itemize}

\subsection{Test Dependencies}
\begin{verbatim}
# Install test dependencies
cargo install cargo-nextest
cargo install cargo-tarpaulin
cargo install cargo-audit
\end{verbatim}

\section{Unit Testing}

\subsection{Component Tests}
\begin{lstlisting}[language=rust]
#[cfg(test)]
mod tests {
    use super::*;

    #[test]
    fn test_firewall_rule_matching() {
        let rule = Rule::new()
            .source("192.168.1.0/24")
            .destination_port(80);
        
        let packet = Packet::new()
            .source("192.168.1.100")
            .destination_port(80);
            
        assert!(rule.matches(&packet));
    }
}
\end{lstlisting}

\subsection{Mock Objects}
\begin{lstlisting}[language=rust]
#[cfg(test)]
mod tests {
    use mockall::*;

    mock! {
        NetworkInterface {
            fn send_packet(&self, packet: &Packet) -> Result<(), Error>;
            fn receive_packet(&self) -> Result<Packet, Error>;
        }
    }
}
\end{lstlisting}

\section{Integration Testing}

\subsection{Component Integration}
\begin{itemize}
    \item Firewall and Network Parser integration
    \item Controller and Firewall integration
    \item Full system integration
\end{itemize}

\subsection{Test Scenarios}
\begin{enumerate}[label=\arabic*.]
    \item Basic packet filtering
    \item Rule application and updates
    \item Performance under load
    \item Error handling and recovery
\end{enumerate}

\section{Performance Testing}

\subsection{Benchmarking}
\begin{lstlisting}[language=rust]
#[bench]
fn bench_packet_processing(b: &mut Bencher) {
    let packet = generate_test_packet();
    b.iter(|| {
        process_packet(&packet)
    });
}
\end{lstlisting}

\subsection{Load Testing}
\begin{verbatim}
# Network throughput test
iperf3 -c target_host -P 4 -t 30

# Rule processing benchmark
cargo bench --bench rule_matching

# System stress test
stress-ng --cpu 8 --io 4 --vm 2 --vm-bytes 1G
\end{verbatim}

\section{Security Testing}

\subsection{Penetration Testing}
\begin{itemize}
    \item Network vulnerability scanning
    \item API security testing
    \item Authentication testing
    \item Access control verification
\end{itemize}

\subsection{Security Compliance Tests}
\begin{itemize}
    \item Input validation
    \item Output sanitization
    \item Encryption verification
    \item Secure communication
\end{itemize}

\section{Automated Testing}

\subsection{CI/CD Integration}
\begin{verbatim}
# Run test suite
cargo test --all-features

# Run security audit
cargo audit

# Generate coverage report
cargo tarpaulin
\end{verbatim}

\subsection{Test Automation}
\begin{itemize}
    \item GitHub Actions workflows
    \item Automated regression testing
    \item Continuous integration
    \item Release testing
\end{itemize}

\section{Test Data Management}

\subsection{Test Data Generation}
\begin{lstlisting}[language=rust]
fn generate_test_packet() -> Packet {
    Packet::builder()
        .source_ip("192.168.1.100")
        .destination_ip("10.0.0.1")
        .source_port(12345)
        .destination_port(80)
        .protocol(Protocol::TCP)
        .build()
}
\end{lstlisting}

\subsection{Test Environments}
\begin{itemize}
    \item Development
    \item Staging
    \item Production-like
    \item Isolated testing
\end{itemize}

\section{Error Testing}

\subsection{Error Scenarios}
\begin{itemize}
    \item Network failures
    \item Resource exhaustion
    \item Invalid configurations
    \item Concurrent access issues
\end{itemize}

\subsection{Error Handling Tests}
\begin{lstlisting}[language=rust]
#[test]
fn test_error_handling() {
    let result = process_invalid_packet();
    assert!(matches!(result, 
        Err(Error::InvalidPacket(_))));
}
\end{lstlisting}

\section{Performance Profiling}

\subsection{CPU Profiling}
\begin{verbatim}
# Run with perf
perf record --call-graph dwarf ./target/release/adam

# Generate flamegraph
perf script | stackcollapse-perf.pl | flamegraph.pl > cpu.svg
\end{verbatim}

\subsection{Memory Profiling}
\begin{itemize}
    \item Memory leak detection
    \item Allocation patterns
    \item Cache utilization
    \item Memory pressure testing
\end{itemize}

\section{Test Reporting}

\subsection{Coverage Reports}
\begin{verbatim}
# Generate HTML coverage report
cargo tarpaulin --out Html

# Generate coverage summary
cargo tarpaulin --out Xml
\end{verbatim}

\subsection{Test Documentation}
\begin{itemize}
    \item Test case documentation
    \item Coverage analysis
    \item Performance benchmarks
    \item Security audit reports
\end{itemize}

\section{Best Practices}

\subsection{Testing Standards}
\begin{itemize}
    \item Test naming conventions
    \item Documentation requirements
    \item Code coverage targets
    \item Performance benchmarks
\end{itemize}

\subsection{Quality Assurance}
\begin{itemize}
    \item Code review process
    \item Test review process
    \item Performance review
    \item Security review
\end{itemize}

\end{document} 
