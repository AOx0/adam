\documentclass{article}
\usepackage{hyperref}

\title{Usage Guide for ADAM}
\author{AOx0}
\date{\today}

\begin{document}

\maketitle

\section{Introduction}
This document provides detailed instructions on how to run the ADAM project. It includes sections on prerequisites, dependencies, and running the code.

\section{Prerequisites}
Before running the code, ensure you have the following prerequisites installed:

\subsection{Cranelift Codegen Backend}
\begin{verbatim}
rustup component add rustc-codegen-cranelift-preview --toolchain nightly
\end{verbatim}

\subsection{Dependencies}
\begin{enumerate}
    \item Install \texttt{lld}: \texttt{paru -S lld} or \texttt{sudo apt-get install lld}
    \item Install \texttt{bpf-linker}: \texttt{cargo install bpf-linker}
    \item Install \texttt{zellij}: \texttt{cargo install zellij}
    \item Install \texttt{cargo-watch}: \texttt{cargo install cargo-watch}
    \item Install \texttt{just}: \texttt{cargo install just}
    \item Install \texttt{hurl}: \texttt{cargo install hurl}
\end{enumerate}

You may install all packages via your package manager, for example, for Arch Linux:
\begin{verbatim}
paru -S just hurl zellij cargo-watch lld
cargo install bpf-linker
\end{verbatim}

\section{Running the Code}
All recipe definitions are available in the \texttt{justfile}.

\subsection{Running Everything}
To run all components, execute:
\begin{verbatim}
just run
\end{verbatim}

You may also specify the firewall wifi interface you want to attach to:
\begin{verbatim}
just run wlan0
\end{verbatim}

\subsection{Running Backend}
To run up to the \texttt{controller}, perform:
\begin{verbatim}
just run-simple
\end{verbatim}

\subsection{Running Frontend}
To run the frontend, perform:
\begin{verbatim}
just run-front-watch
\end{verbatim}

\section{Firewall Component}
The firewall component has additional instructions for building and running.

\subsection{Prerequisites}
Install \texttt{bpf-linker}:
\begin{verbatim}
cargo install bpf-linker
\end{verbatim}

\subsection{Build eBPF}
\begin{verbatim}
cargo xtask build-ebpf <NAME>
\end{verbatim}
To perform a release build, use the \texttt{--release} flag. You may also change the target architecture with the \texttt{--target} flag.

\subsection{Build Userspace}
\begin{verbatim}
cargo build
\end{verbatim}

\subsection{Build eBPF and Userspace}
\begin{verbatim}
cargo xtask build <NAME>
\end{verbatim}

\subsection{Run}
\begin{verbatim}
RUST_LOG=info cargo xtask run <NAME>
\end{verbatim}

\section{Building the Project}
To build the entire project, use the following command:
\begin{verbatim}
cargo build
\end{verbatim}

\section{Testing the Project}
To run tests for the project, use the following command:
\begin{verbatim}
cargo test
\end{verbatim}

\section{Detailed Explanations and Examples}
\subsection{Example 1: Running the Firewall Component}
Here is a step-by-step example of running the firewall component:
\begin{enumerate}
    \item Install the prerequisites as mentioned in the prerequisites section.
    \item Build the eBPF component:
    \begin{verbatim}
    cargo xtask build-ebpf firewall
    \end{verbatim}
    \item Build the userspace component:
    \begin{verbatim}
    cargo build
    \end{verbatim}
    \item Run the firewall component:
    \begin{verbatim}
    RUST_LOG=info cargo xtask run firewall
    \end{verbatim}
\end{enumerate}

\subsection{Example 2: Running the Entire Project}
Here is a step-by-step example of running the entire project:
\begin{enumerate}
    \item Install the prerequisites as mentioned in the prerequisites section.
    \item Run all components:
    \begin{verbatim}
    just run
    \end{verbatim}
    \item Specify the firewall wifi interface if needed:
    \begin{verbatim}
    just run wlan0
    \end{verbatim}
\end{enumerate}

\subsection{Example 3: Running the Frontend}
Here is a step-by-step example of running the frontend:
\begin{enumerate}
    \item Install the prerequisites as mentioned in the prerequisites section.
    \item Run the frontend:
    \begin{verbatim}
    just run-front-watch
    \end{verbatim}
\end{enumerate}

\end{document}
