\documentclass{article}
\usepackage{hyperref}

\title{Advanced Usage Guide for ADAM}
\author{AOx0}
\date{\today}

\begin{document}

\maketitle

\section{Introduction}
This document provides advanced usage instructions for the ADAM project. It includes sections on advanced configuration, customization, and examples of advanced usage scenarios.

\section{Advanced Configuration}
\subsection{Customizing the Firewall Component}
The firewall component can be customized by modifying the configuration files located in the `firewall/config` directory. Here are some common customizations:

\subsubsection{Changing the Default Rules}
To change the default rules, edit the `firewall/config/rules.toml` file. For example, to allow traffic from a specific IP address, add the following rule:
\begin{verbatim}
[[rules]]
action = "allow"
source = "192.168.1.100"
\end{verbatim}

\subsubsection{Modifying the Logging Level}
To modify the logging level, edit the `firewall/config/logging.toml` file. For example, to set the logging level to `debug`, change the following line:
\begin{verbatim}
level = "debug"
\end{verbatim}

\subsection{Advanced Network Configuration}
The ADAM project supports advanced network configurations. Here are some examples:

\subsubsection{Configuring Multiple Network Interfaces}
To configure multiple network interfaces, edit the `network/interfaces.toml` file. For example, to add a new network interface, add the following section:
\begin{verbatim}
[[interfaces]]
name = "eth1"
address = "192.168.2.1"
netmask = "255.255.255.0"
\end{verbatim}

\subsubsection{Setting Up VLANs}
To set up VLANs, edit the `network/vlans.toml` file. For example, to create a new VLAN, add the following section:
\begin{verbatim}
[[vlans]]
id = 10
name = "VLAN10"
interface = "eth0"
\end{verbatim}

\section{Advanced Customization}
\subsection{Customizing the Frontend}
The frontend can be customized by modifying the files located in the `front` directory. Here are some common customizations:

\subsubsection{Changing the Theme}
To change the theme, edit the `front/static/styles.css` file. For example, to change the background color, modify the following line:
\begin{verbatim}
body {
    background-color: #f0f0f0;
}
\end{verbatim}

\subsubsection{Adding New Pages}
To add new pages, create a new file in the `front/pages` directory. For example, to add a new page called `about.html`, create the file `front/pages/about.html` and add the following content:
\begin{verbatim}
<!DOCTYPE html>
<html>
<head>
    <title>About</title>
</head>
<body>
    <h1>About</h1>
    <p>This is the about page.</p>
</body>
</html>
\end{verbatim}

\section{Advanced Usage Scenarios}
\subsection{Scenario 1: Setting Up a VPN}
Here is a step-by-step example of setting up a VPN using the ADAM project:
\begin{enumerate}
    \item Install the prerequisites as mentioned in the prerequisites section of the usage guide.
    \item Configure the VPN server by editing the `vpn/server.toml` file. For example, to set the server address, modify the following line:
    \begin{verbatim}
    address = "10.0.0.1"
    \end{verbatim}
    \item Start the VPN server:
    \begin{verbatim}
    just run-vpn-server
    \end{verbatim}
    \item Configure the VPN client by editing the `vpn/client.toml` file. For example, to set the client address, modify the following line:
    \begin{verbatim}
    address = "10.0.0.2"
    \end{verbatim}
    \item Start the VPN client:
    \begin{verbatim}
    just run-vpn-client
    \end{verbatim}
\end{enumerate}

\subsection{Scenario 2: Setting Up a Load Balancer}
Here is a step-by-step example of setting up a load balancer using the ADAM project:
\begin{enumerate}
    \item Install the prerequisites as mentioned in the prerequisites section of the usage guide.
    \item Configure the load balancer by editing the `load_balancer/config.toml` file. For example, to add backend servers, add the following section:
    \begin{verbatim}
    [[backends]]
    address = "192.168.1.101"
    port = 80

    [[backends]]
    address = "192.168.1.102"
    port = 80
    \end{verbatim}
    \item Start the load balancer:
    \begin{verbatim}
    just run-load-balancer
    \end{verbatim}
\end{enumerate}

\end{document}
